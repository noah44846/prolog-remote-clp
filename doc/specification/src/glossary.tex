\newglossaryentry{esp8266}
{
    name=ESP8266,
    description={Circuit intégré à microcontrôleur avec connexion Wi-Fi développé par le fabricant chinois Espressif}
}

\newglossaryentry{heroku}
{
    name=Heroku,
    description={Heroku est une plate-forme en tant que services (\acrshort{paas}) qui permet aux développeurs de créer, exécuter et exploiter des applications entièrement dans le \emph{cloud}}
}

%% Pas plutôt un bas de page ?
\newglossaryentry{topic}
{
    name=\emph{topic},
    description={Arborescence permettant d'accéder à une ressource, utilisé par le protocol MQTT}
}

\newglossaryentry{docker}
{
    name=Docker,
    description={Docker est une plateforme de conteneurisation permettant aux développeurs de déployer du code ainsi que ses dépendances associées de manière simple et sans se soucier du système d'exploitation cible}
}

\newglossaryentry{serverless}
{
    name={\emph{Serverless computing}},
    description={Le \emph{serverless computing} est un paradigme de \emph{cloud computing} dans lequel le fournisseur de serveur gère dynamiquement les ressources allouées au service client. Le prix dépend des ressources effectivement consommées et non des capacités d'un serveur acheté à l'avance \cite{wiki-serverless}}
}

\newglossaryentry{cloud}
{
    name=\emph{cloud},
    description={Le \emph{cloud} consiste en des serveurs informatiques distants communiquants par l'intermédiaire d'un réseau, généralement Internet, pour stocker des données ou les exploiter \cite{wiki-cloud}}
}

\newglossaryentry{fog-computing}
{
    name={\emph{fog computing}},
    description={Le \emph{fog computing} consiste à exploiter des applications et des infrastructures de traitement et de stockage de proximité, servant d'intermédiaire entre des objets connectés et une architecture de \emph{cloud computing} classique \cite{wiki-fog-computing}}
}

\newglossaryentry{thing}
{
    name={\emph{thing}},
    description={Un \emph{thing} est une représentation d'un dispositif ou d'une entité logique spécifique. Il peut s'agir d'un dispositif ou d'un capteur physique. Dans notre projet, un \emph{thing} représente un \uc{}.}
}

\newglossaryentry{cicd}
{
    name={CI/CD},
    description={De l'anglais \emph{Continuous Integration/Continuous Delivery or Deployement} est un concept dans le développement logiciel qui vise à automatiser les tests et le déploiement du produit afin de minimiser le temps entre les nouvelles versions et de prévenir les erreurs dans le code. Souvent le CI/CD consiste à tester le code automatiquement lors de chaque commit et de le déployer lors d'un \emph{push} sur une certaine branche.}
}