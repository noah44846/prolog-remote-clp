%%%%%%%%%%%%%%%%%%%%%%%%%%%%%%%%%%%%%%%%%%%%%%%%%%%%%%%%%%%%%%%%%%%%%%%%%%%%%%%%
% School of Engineering Fribourg - LaTeX Template                              %
% Florent Kilchoer - 2020                                                      %
% Special Thanks to Yannis Huber for theme inspiration                         %
%                                                                              %
% THIS IS NOT AN OFFICIAL TEMPLATE                                             %
%%%%%%%%%%%%%%%%%%%%%%%%%%%%%%%%%%%%%%%%%%%%%%%%%%%%%%%%%%%%%%%%%%%%%%%%%%%%%%%%

\documentclass[en]{customTemplate}

%% Glossary & Acronyms
\makeglossaries

\newglossaryentry{esp8266}
{
    name=ESP8266,
    description={Circuit intégré à microcontrôleur avec connexion Wi-Fi développé par le fabricant chinois Espressif}
}

\newglossaryentry{heroku}
{
    name=Heroku,
    description={Heroku est une plate-forme en tant que services (\acrshort{paas}) qui permet aux développeurs de créer, exécuter et exploiter des applications entièrement dans le \emph{cloud}}
}

%% Pas plutôt un bas de page ?
\newglossaryentry{topic}
{
    name=\emph{topic},
    description={Arborescence permettant d'accéder à une ressource, utilisé par le protocol MQTT}
}

\newglossaryentry{docker}
{
    name=Docker,
    description={Docker est une plateforme de conteneurisation permettant aux développeurs de déployer du code ainsi que ses dépendances associées de manière simple et sans se soucier du système d'exploitation cible}
}

\newglossaryentry{serverless}
{
    name={\emph{Serverless computing}},
    description={Le \emph{serverless computing} est un paradigme de \emph{cloud computing} dans lequel le fournisseur de serveur gère dynamiquement les ressources allouées au service client. Le prix dépend des ressources effectivement consommées et non des capacités d'un serveur acheté à l'avance \cite{wiki-serverless}}
}

\newglossaryentry{cloud}
{
    name=\emph{cloud},
    description={Le \emph{cloud} consiste en des serveurs informatiques distants communiquants par l'intermédiaire d'un réseau, généralement Internet, pour stocker des données ou les exploiter \cite{wiki-cloud}}
}

\newglossaryentry{fog-computing}
{
    name={\emph{fog computing}},
    description={Le \emph{fog computing} consiste à exploiter des applications et des infrastructures de traitement et de stockage de proximité, servant d'intermédiaire entre des objets connectés et une architecture de \emph{cloud computing} classique \cite{wiki-fog-computing}}
}

\newglossaryentry{thing}
{
    name={\emph{thing}},
    description={Un \emph{thing} est une représentation d'un dispositif ou d'une entité logique spécifique. Il peut s'agir d'un dispositif ou d'un capteur physique. Dans notre projet, un \emph{thing} représente un \uc{}.}
}

\newglossaryentry{cicd}
{
    name={CI/CD},
    description={De l'anglais \emph{Continuous Integration/Continuous Delivery or Deployement} est un concept dans le développement logiciel qui vise à automatiser les tests et le déploiement du produit afin de minimiser le temps entre les nouvelles versions et de prévenir les erreurs dans le code. Souvent le CI/CD consiste à tester le code automatiquement lors de chaque commit et de le déployer lors d'un \emph{push} sur une certaine branche.}
}
%% ACRONYMS
\newacronym{iot}{IoT}{Internet des Objets}
\newacronym{gke}{GKE}{Google Kubernetes Engine}
\newacronym{http}{HTTP}{Hypertext Transfer Protocol}
\newacronym{mqtt}{MQTT}{Message Queuing Telemetry Transport}
\newacronym{lipo}{LiPo}{Lithium Polymer}
\newacronym{ap}{AP}{Access Point}
\newacronym{dns}{DNS}{Domain Name System}
\newacronym{dhcp}{DHCP}{Dynamic Host Configuration Protocol}
\newacronym{ftp}{FTP}{File Transfert Protocol}
\newacronym{paas}{PaaS}{Platform as a service}
\newacronym{aws}{AWS}{Amazon Web Services}
\newacronym{oidc}{OIDC}{OpenID Connect}
\newacronym{jwt}{JWT}{JSON Web Token}
\newacronym{lwt}{LWT}{Last Will and Testament}
\newacronym{orm}{ORM}{Object-relational mapping}
\newacronym{rest}{REST}{Representational state transfer}

%% REFERENCES
\addbibresource{sources.bib}

%% REVISIONS
\AddRevision{0.1.0}{4 March 2024}{Initial prototype}
\AddRevision{0.2.0}{10 March 2024}{Revised prototype}
\AddRevision{1.0.0}{17 March 2024}{Final version}

%%%%%%%%%%%%%%%%%%%%%%%%%%%%%%%%%%%%%%%%%%%%%%%%%%%%%%%%%%%%%%%%%%%%%%%%%%%%%%%%

% Comment these lines to hide informations
%% DOCUMENT INFORMATION
\title{Prolog Remote Constraint Logic Programming}
\subtitle{Specifications}
%\course{Course}
\students{Noah Godel}{}{}
\supervisors{Frédéric Bapst}{}


% Uncomment this line to use custom/no date (this replace the last revision date)
%\lastReleaseDate{\today}
%\lastReleaseDate{\empty}

\usepackage{xifthen}% provides \isempty test

\begin{document}


%%%%%%%%%%%%%%%%%%%%%%%%%%%%%%%%%%% DOCUMENT %%%%%%%%%%%%%%%%%%%%%%%%%%%%%%%%%%% 
\maketitlepage{}
% OPTIONAL : Display the revision table
\makerevisiontable{}

% OPTIONAL : The abstract has to be written in abstract.tex
%\input{abstract}

% OPTIONAL : Uncomment the following line to add a table of content
\maketableofcontent{}

%\smallheader
\fullheader
%\noheader

\pagenumbering{arabic}

%\begin{comment}
\section{Context}

The purpose of this project is to develop an innovative solution that facilitates the integration of Operations Research (OR) libraries, such as Google OR Tools and Gecode, as a backend for Constraint Logic Programming (CLP) in Prolog. The primary goal is twofold. On the one side it is to create a Web API that allows multiple users to access and utilize these OR libraries simultaneously. On the other side it is to create Prolog client that connects to this API and uses it. To achieve scalability and resource optimization, the Web API will be deployed on a Kubernetes cluster.\\

In a previous attempt to bridge the gap between Prolog and the Google OR Tools library, a project was initiated. However, this approach required the local installation of the library, presenting limitations in terms of accessibility. To overcome these challenges, our current initiative focuses on developing a Web-based solution that eliminates the need for local installations, giving us an easy-to-use library for CLP in Prolog. This approach not only enhances accessibility but in addition promotes efficient utilization of resources, making it an ideal choice for collaborative and concurrent CLP programming tasks. The proposed architecture leverages the power of Kubernetes to provide a scalable and resilient environment for hosting the Web API, making it well-suited for deployment in various operational scenarios.\\

In the context of this project we will focus on the implementation of an API for the Google OR Tools library. In the future, the same approach could be extended to other OR libraries, such as Gecode.

%\myFigure{vrp.svg}{width=0.6\textwidth}{Example of the VRP}{vrp}

\clearpage

\section{Goals}

The primary goals of this project are the following:

\begin{itemize}
    \item Implement a Web API that provides access to part of the CP-SAT solver of the Google OR Tools library for Constraint Logic Programming in Prolog. The API should be designed to handle multiple concurrent requests and provide some scalability in case of increased usage in the future. A token-based authentication mechanism should be implemented to ensure secure access to the API.
    \item Deploy the Web API on a Kubernetes cluster to ensure scalability and resilience. The deployment should be automated with the use of Gitlab CI/CD pipelines.
    \item Develop an OS independent client library for SWI-Prolog that allows easy access to the Web API. The client library should be usable in a way similar to other CLP libraries in Prolog, such as clpfd.
    \item Perform a series of tests to ensure the reliability and performance of the Web API. The tests should include unit tests and performance tests.
    \item A series of demonstration programs should be implemented to showcase the capabilities of the Web API and the client library.
\end{itemize}

If the planned tasks are completed faster than expected, the addition of a second OR library, such as Gecode, to the Web API could be considered. Alternatively the use of more features of the Google OR Tools library could be implemented. Another option would be to implement a client in a different programming language, such as Java. Another possible extension would be to implement the client library for other Prolog systems, such as GNU Prolog.

\clearpage

\section{Tasks}

To achieve the above-mentioned goals the following tasks need to be carried out:

\begin{itemize}
    \item Analysis of possible architecture for the Web API and the client library.
    \item Design of the Web API and the client library.
    \item Design of the client library for Prolog and demonstration programs. 
    \item Implementation of basic endpoints for the Web API to have a basic working API.
    \item Implementation of the client library for Prolog.
    \item Implementation of the remaining endpoints for the Web API.
    \item Implementation of the authentication mechanism for the Web API.
    \item Deployment of the Web API on a Kubernetes cluster and Gitlab CI/CD pipeline setup.
    \item Performance testing of the Web API.
\end{itemize}

\section{Planning}   

See next page.

% remove header & footer
\fancyhf{}
\renewcommand{\headrulewidth}{0pt}

\begin{sidewaysfigure}
   \myFigure{gantt.png}{width=\textwidth}{Gantt diagram of the planning}{planning}
\end{sidewaysfigure}

\printglossary[type=\acronymtype]
\printglossary
\printbibliography{}
\end{document}